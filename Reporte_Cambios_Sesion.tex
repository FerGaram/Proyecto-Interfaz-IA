\documentclass{article}
\usepackage[utf8]{inputenc}
\usepackage{geometry}
\usepackage{listings}
\usepackage{color}
\usepackage{longtable}
\geometry{margin=1in}
\definecolor{codegray}{rgb}{0.5,0.5,0.5}
\definecolor{backcolour}{rgb}{0.95,0.95,0.92}

\lstdefinestyle{mystyle}{
    backgroundcolor=\color{backcolour},
    commentstyle=\color{codegray},
    keywordstyle=\color{blue},
    numberstyle=\tiny\color{codegray},
    stringstyle=\color{red},
    basicstyle=\ttfamily\footnotesize,
    breaklines=true,
    captionpos=b,
    keepspaces=true,
    numbers=left,
    numbersep=5pt,
    showspaces=false,
    showstringspaces=false,
    showtabs=false,
    tabsize=2
}
\lstset{style=mystyle}

\title{Reporte de Cambios - Sesión de Desarrollo}
\author{Proyecto Interfaz IA}
\date{15 de junio de 2025}

\begin{document}

\maketitle

\section*{Resumen de Cambios}
Durante esta sesión de desarrollo se realizaron mejoras significativas en la interfaz y experiencia de usuario del proyecto, destacando:
\begin{itemize}
    \item Integración de Material UI para botones principales.
    \item Implementación de atajos de teclado (Ctrl+Z, Ctrl+C, Ctrl+V, Ctrl+X, Ctrl+D) para manipulación de nodos.
    \item Visualización de los atajos en el panel lateral derecho superior.
    \item Refactorización para compatibilidad con React Flow y mejores prácticas de código.
\end{itemize}

\section*{Fragmentos de Código Relevantes}
\subsection*{1. Integración de Atajos de Teclado}
\textbf{Archivo: src/components/KeyboardShortcuts.tsx}
\begin{lstlisting}[language=JavaScript, caption=KeyboardShortcuts.tsx]
import { useEffect } from "react";

interface KeyboardShortcutsProps {
  onUndo: () => void;
  onCopy: () => void;
  onPaste: () => void;
  onCut: () => void;
  onDuplicate: () => void;
}

export const KeyboardShortcuts = ({
  onUndo,
  onCopy,
  onPaste,
  onCut,
  onDuplicate,
}: KeyboardShortcutsProps) => {
  useEffect(() => {
    const handleKeyDown = (e: KeyboardEvent) => {
      if (e.ctrlKey) {
        switch (e.key.toLowerCase()) {
          case "z":
            e.preventDefault();
            onUndo();
            break;
          case "c":
            e.preventDefault();
            onCopy();
            break;
          case "v":
            e.preventDefault();
            onPaste();
            break;
          case "x":
            e.preventDefault();
            onCut();
            break;
          case "d":
            e.preventDefault();
            onDuplicate();
            break;
          default:
            break;
        }
      }
    };
    window.addEventListener("keydown", handleKeyDown);
    return () => window.removeEventListener("keydown", handleKeyDown);
  }, [onUndo, onCopy, onPaste, onCut, onDuplicate]);
  return null;
};
\end{lstlisting}

\subsection*{2. Lógica de Atajos y Portapapeles}
\textbf{Archivo: src/App.tsx}
\begin{lstlisting}[language=JavaScript, caption=Fragmento de integración de atajos en App.tsx]
// ...existing code...
const getSelectedNodes = () => nodes.filter((n: any) => n.selected);

const handleCopy = useCallback(() => {
  clipboard.current = getSelectedNodes();
}, [nodes]);

const handlePaste = useCallback(() => {
  if (clipboard.current && clipboard.current.length > 0) {
    saveToHistory();
    const newNodes = clipboard.current.map((n) => ({
      ...n,
      id: (ultimoId.current + 1).toString(),
      position: { x: n.position.x + 40, y: n.position.y + 40 },
      selected: false,
    }));
    ultimoId.current += newNodes.length;
    setNodes((nds) => [...nds, ...newNodes]);
  }
}, [saveToHistory]);

const handleCut = useCallback(() => {
  saveToHistory();
  clipboard.current = getSelectedNodes();
  setNodes((nds) => nds.filter((n: any) => !n.selected));
}, [nodes, saveToHistory]);

const handleDuplicate = useCallback(() => {
  const selected = getSelectedNodes();
  if (selected.length > 0) {
    saveToHistory();
    const newNodes = selected.map((n) => ({
      ...n,
      id: (ultimoId.current + 1).toString(),
      position: { x: n.position.x + 40, y: n.position.y + 40 },
      selected: false,
    }));
    ultimoId.current += newNodes.length;
    setNodes((nds) => [...nds, ...newNodes]);
  }
}, [nodes, saveToHistory]);
// ...existing code...
\end{lstlisting}

\subsection*{3. Visualización de Atajos en el Panel Lateral}
\textbf{Archivo: src/App.tsx}
\begin{lstlisting}[language=JavaScript, caption=Panel lateral con atajos]
<div style={{ fontSize: '0.75rem', color: '#eee', marginTop: 8 }}>
  <b>Atajos útiles:</b>
  <ul style={{ margin: 0, paddingLeft: 18 }}>
    <li><b>Ctrl+Z</b>: Deshacer</li>
    <li><b>Ctrl+C</b>: Copiar nodo(s) seleccionado(s)</li>
    <li><b>Ctrl+V</b>: Pegar nodo(s)</li>
    <li><b>Ctrl+X</b>: Cortar nodo(s)</li>
    <li><b>Ctrl+D</b>: Duplicar nodo(s)</li>
  </ul>
</div>
\end{lstlisting}

\section*{Tabla de Cambios en App.tsx}
\begin{longtable}{|p{3cm}|p{4cm}|p{4cm}|p{4cm}|}
\hline
\textbf{Diferencia} & \textbf{App.tsx (anterior)} & \textbf{App.tsx (actualizado)} & \textbf{¿Por qué se hizo el cambio?} \\
\hline
Estilo de comillas & Comillas simples ' & Comillas dobles " & Estilo estándar en ESLint/Prettier para JS/TS moderno \\
\hline
Uso de punto y coma & No se usan & Se usan ; & Mejora consistencia y previene errores con ASI \\
\hline
setNodoInicial y setNodoFinal & Declarados pero no usados & Solo [nodoInicial] y [nodoFinal] (sin setters) & Se eliminaron setters innecesarios para simplificar código \\
\hline
Integración de Material UI & Botones HTML nativos & Botones de Material UI (Button, Stack) & Mejorar presentación visual y experiencia de usuario \\
\hline
Atajos de teclado & No existen & Se agregan handlers y componente KeyboardShortcuts & Mejorar accesibilidad y productividad del usuario \\
\hline
Visualización de atajos & No existen & Se muestran en el panel lateral derecho superior & Hacer visibles los atajos para el usuario \\
\hline
Compatibilidad con React Flow & Selección manual o no soportada & Uso de propiedad selected de React Flow & Mantener compatibilidad y mejores prácticas \\
\hline
\end{longtable}

\end{document}
