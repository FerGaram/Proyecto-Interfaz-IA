\documentclass{article}
\usepackage[utf8]{inputenc}
\usepackage{geometry}
\usepackage{listings}
\usepackage{color}
\usepackage{longtable}
\geometry{margin=1in}
\definecolor{codegray}{rgb}{0.5,0.5,0.5}
\definecolor{backcolour}{rgb}{0.95,0.95,0.92}

\lstdefinestyle{mystyle}{
    backgroundcolor=\color{backcolour},
    commentstyle=\color{codegray},
    keywordstyle=\color{blue},
    numberstyle=\tiny\color{codegray},
    stringstyle=\color{red},
    basicstyle=\ttfamily\footnotesize,
    breaklines=true,
    captionpos=b,
    keepspaces=true,
    numbers=left,
    numbersep=5pt,
    showspaces=false,
    showstringspaces=false,
    showtabs=false,
    tabsize=2
}
\lstset{style=mystyle}

\title{Reporte de Cambios - Sesión de Desarrollo}
\author{Jose Ramon Aragon Toledo}
\date{15 de junio de 2025}

\begin{document}

\maketitle

\section*{Resumen de Cambios}
Durante esta sesión de desarrollo se realizaron mejoras significativas en la interfaz y experiencia de usuario del proyecto, destacando:
\begin{itemize}
    \item Integración de Material UI para botones principales.
    \item Implementación de atajos de teclado (Ctrl+Z, Ctrl+C, Ctrl+V, Ctrl+X, Ctrl+D) para manipulación de nodos.
    \item Visualización de los atajos en el panel lateral derecho superior.
    \item Refactorización para compatibilidad con React Flow y mejores prácticas de código.
\end{itemize}

\section*{Fragmentos de Código Relevantes}
\subsection*{1. Integración de Atajos de Teclado}
\textbf{Archivo: src/components/KeyboardShortcuts.tsx}
\begin{lstlisting}[language=JavaScript, caption=KeyboardShortcuts.tsx]
import { useEffect } from "react";

interface KeyboardShortcutsProps {
  onUndo: () => void;
  onCopy: () => void;
  onPaste: () => void;
  onCut: () => void;
  onDuplicate: () => void;
}

export const KeyboardShortcuts = ({
  onUndo,
  onCopy,
  onPaste,
  onCut,
  onDuplicate,
}: KeyboardShortcutsProps) => {
  useEffect(() => {
    const handleKeyDown = (e: KeyboardEvent) => {
      if (e.ctrlKey) {
        switch (e.key.toLowerCase()) {
          case "z":
            e.preventDefault();
            onUndo();
            break;
          case "c":
            e.preventDefault();
            onCopy();
            break;
          case "v":
            e.preventDefault();
            onPaste();
            break;
          case "x":
            e.preventDefault();
            onCut();
            break;
          case "d":
            e.preventDefault();
            onDuplicate();
            break;
          default:
            break;
        }
      }
    };
    window.addEventListener("keydown", handleKeyDown);
    return () => window.removeEventListener("keydown", handleKeyDown);
  }, [onUndo, onCopy, onPaste, onCut, onDuplicate]);
  return null;
};
\end{lstlisting}

\subsection*{2. Lógica de Atajos y Portapapeles}
\textbf{Archivo: src/App.tsx}
\begin{lstlisting}[language=JavaScript, caption=Fragmento de integración de atajos en App.tsx]
// ...existing code...
const getSelectedNodes = () => nodes.filter((n: any) => n.selected);

const handleCopy = useCallback(() => {
  clipboard.current = getSelectedNodes();
}, [nodes]);

const handlePaste = useCallback(() => {
  if (clipboard.current && clipboard.current.length > 0) {
    saveToHistory();
    let idCounter = ultimoId.current;
    const newNodes = clipboard.current.map((n) => {
      idCounter++;
      return {
        ...n,
        id: idCounter.toString(),
        position: { x: n.position.x + 40, y: n.position.y + 40 },
        selected: false,
      };
    });
    ultimoId.current = idCounter;
    setNodes((nds) => [...nds, ...newNodes]);
  }
}, [saveToHistory]);

const handleCut = useCallback(() => {
  saveToHistory();
  clipboard.current = getSelectedNodes();
  setNodes((nds) => nds.filter((n: any) => !n.selected));
}, [nodes, saveToHistory]);

const handleDuplicate = useCallback(() => {
  const selected = getSelectedNodes();
  if (selected.length > 0) {
    saveToHistory();
    const newNodes = selected.map((n) => ({
      ...n,
      id: (ultimoId.current + 1).toString(),
      position: { x: n.position.x + 40, y: n.position.y + 40 },
      selected: false,
    }));
    ultimoId.current += newNodes.length;
    setNodes((nds) => [...nds, ...newNodes]);
  }
}, [nodes, saveToHistory]);
// ...existing code...
\end{lstlisting}

\subsection*{3. Visualización de Atajos en el Panel Lateral}
\textbf{Archivo: src/App.tsx}
\begin{lstlisting}[language=JavaScript, caption=Panel lateral con atajos]
<div style={{ fontSize: '0.75rem', color: '#eee', marginTop: 8 }}>
  <b>Atajos útiles:</b>
  <ul style={{ margin: 0, paddingLeft: 18 }}>
    <li><b>Ctrl+Z</b>: Deshacer</li>
    <li><b>Ctrl+C</b>: Copiar nodo(s) seleccionado(s)</li>
    <li><b>Ctrl+V</b>: Pegar nodo(s)</li>
    <li><b>Ctrl+X</b>: Cortar nodo(s)</li>
    <li><b>Ctrl+D</b>: Duplicar nodo(s)</li>
  </ul>
</div>
\end{lstlisting}

\section*{Mejoras de Estética y Experiencia}
Se unificó la estética de todos los botones, listas desplegables y paneles principales:
\begin{itemize}
    \item Todos los botones y dropdowns tienen el mismo tamaño, separación y están alineados y centrados.
    \item Se aplicó fondo transparente con efecto blur a los paneles y controles principales.
    \item Los colores de los botones se adaptan al modo claro/oscuro y a la función de cada uno, pero mantienen coherencia visual.
    \item El panel de controles (post-it) ahora tiene mejor contraste y legibilidad en modo oscuro.
    \item Se corrigió el fondo de los controles nativos de React Flow para que no cambien de color.
\end{itemize}

\subsection*{Fragmento: Unificación de botones y panel superior}
\textbf{Archivo: src/App.tsx}
\begin{lstlisting}[language=JavaScript, caption=Panel superior con botones unificados]
<Panel position="top-center">
  <Box sx={{
    display: 'flex', alignItems: 'center', justifyContent: 'center', gap: 2,
    bgcolor: darkMode ? 'rgba(30,30,40,0.55)' : 'rgba(220,220,230,0.75)',
    backdropFilter: 'blur(8px)', borderRadius: '12px', boxShadow: '0 2px 12px rgba(0,0,0,0.10)',
    padding: '16px 32px', margin: '0 auto', minWidth: 600, maxWidth: '90vw',
  }}>
    <NodeTypeDropdown ... />
    <Stack direction="row" spacing={2} alignItems="center">
      <Button ...>AÑADIR NUEVO NODO</Button>
      <Button ...>LIMPIAR PANTALLA</Button>
      <Button ...>IMPRIMIR NODOS Y ARISTAS</Button>
      <Button ...>{darkMode ? 'MODO CLARO' : 'MODO OSCURO'}</Button>
    </Stack>
  </Box>
</Panel>
\end{lstlisting}

\subsection*{Corrección de IDs únicos al duplicar/copiar/pegar}
\textbf{Archivo: src/App.tsx}
\begin{lstlisting}[language=JavaScript, caption=IDs únicos en duplicado y pegado]
const handlePaste = useCallback(() => {
  if (clipboard.current && clipboard.current.length > 0) {
    saveToHistory();
    let idCounter = ultimoId.current;
    const newNodes = clipboard.current.map((n) => {
      idCounter++;
      return {
        ...n,
        id: idCounter.toString(),
        position: { x: n.position.x + 40, y: n.position.y + 40 },
        selected: false,
      };
    });
    ultimoId.current = idCounter;
    setNodes((nds) => [...nds, ...newNodes]);
  }
}, [saveToHistory]);
\end{lstlisting}

\subsection*{Ajuste de contraste en panel de controles}
\textbf{Archivo: src/App.tsx}
\begin{lstlisting}[language=JavaScript, caption=Panel de controles con fondo adaptativo]
<Box
  className={`postit${minimized ? " minimized" : ""}`}
  ...
  sx={{
    bgcolor: darkMode ? 'rgba(40,40,40,0.96)' : 'background.paper',
    ...
  }}
>
\end{lstlisting}

\section*{Tabla de Cambios Ampliada}
\begin{longtable}{|p{3cm}|p{4cm}|p{4cm}|p{4cm}|}
\hline
\textbf{Diferencia} & \textbf{Antes} & \textbf{Después} & \textbf{¿Por qué se hizo el cambio?} \\
\hline
Unificación de botones y paneles & Botones y paneles con estilos variados & Todos con mismo tamaño, alineación, efecto blur y colores adaptativos & Mejorar coherencia visual y experiencia de usuario \\
\hline
IDs únicos en duplicado/pegado & IDs duplicados al copiar/duplicar varios nodos & IDs únicos e incrementales para cada nodo nuevo & Evitar conflictos y nodos invisibles \\
\hline
Contraste en panel de controles & Fondo opaco dificultaba lectura en modo oscuro & Fondo más sólido y texto legible en ambos modos & Mejorar accesibilidad y legibilidad \\
\hline
Colores adaptativos & Colores fijos en modo claro/oscuro & Colores y fondos que se adaptan al tema & Consistencia visual y accesibilidad \\
\hline
\end{longtable}

\end{document}
